\documentclass[]{article}
\usepackage{lmodern}
\usepackage{amssymb,amsmath}
\usepackage{ifxetex,ifluatex}
\usepackage{fixltx2e} % provides \textsubscript
\ifnum 0\ifxetex 1\fi\ifluatex 1\fi=0 % if pdftex
  \usepackage[T1]{fontenc}
  \usepackage[utf8]{inputenc}
\else % if luatex or xelatex
  \ifxetex
    \usepackage{mathspec}
  \else
    \usepackage{fontspec}
  \fi
  \defaultfontfeatures{Ligatures=TeX,Scale=MatchLowercase}
\fi
% use upquote if available, for straight quotes in verbatim environments
\IfFileExists{upquote.sty}{\usepackage{upquote}}{}
% use microtype if available
\IfFileExists{microtype.sty}{%
\usepackage{microtype}
\UseMicrotypeSet[protrusion]{basicmath} % disable protrusion for tt fonts
}{}
\usepackage[margin=1in]{geometry}
\usepackage{hyperref}
\hypersetup{unicode=true,
            pdftitle={Capital Bikeshare System Analysis},
            pdfauthor={Ross Bradley},
            pdfborder={0 0 0},
            breaklinks=true}
\urlstyle{same}  % don't use monospace font for urls
\usepackage{color}
\usepackage{fancyvrb}
\newcommand{\VerbBar}{|}
\newcommand{\VERB}{\Verb[commandchars=\\\{\}]}
\DefineVerbatimEnvironment{Highlighting}{Verbatim}{commandchars=\\\{\}}
% Add ',fontsize=\small' for more characters per line
\usepackage{framed}
\definecolor{shadecolor}{RGB}{248,248,248}
\newenvironment{Shaded}{\begin{snugshade}}{\end{snugshade}}
\newcommand{\KeywordTok}[1]{\textcolor[rgb]{0.13,0.29,0.53}{\textbf{#1}}}
\newcommand{\DataTypeTok}[1]{\textcolor[rgb]{0.13,0.29,0.53}{#1}}
\newcommand{\DecValTok}[1]{\textcolor[rgb]{0.00,0.00,0.81}{#1}}
\newcommand{\BaseNTok}[1]{\textcolor[rgb]{0.00,0.00,0.81}{#1}}
\newcommand{\FloatTok}[1]{\textcolor[rgb]{0.00,0.00,0.81}{#1}}
\newcommand{\ConstantTok}[1]{\textcolor[rgb]{0.00,0.00,0.00}{#1}}
\newcommand{\CharTok}[1]{\textcolor[rgb]{0.31,0.60,0.02}{#1}}
\newcommand{\SpecialCharTok}[1]{\textcolor[rgb]{0.00,0.00,0.00}{#1}}
\newcommand{\StringTok}[1]{\textcolor[rgb]{0.31,0.60,0.02}{#1}}
\newcommand{\VerbatimStringTok}[1]{\textcolor[rgb]{0.31,0.60,0.02}{#1}}
\newcommand{\SpecialStringTok}[1]{\textcolor[rgb]{0.31,0.60,0.02}{#1}}
\newcommand{\ImportTok}[1]{#1}
\newcommand{\CommentTok}[1]{\textcolor[rgb]{0.56,0.35,0.01}{\textit{#1}}}
\newcommand{\DocumentationTok}[1]{\textcolor[rgb]{0.56,0.35,0.01}{\textbf{\textit{#1}}}}
\newcommand{\AnnotationTok}[1]{\textcolor[rgb]{0.56,0.35,0.01}{\textbf{\textit{#1}}}}
\newcommand{\CommentVarTok}[1]{\textcolor[rgb]{0.56,0.35,0.01}{\textbf{\textit{#1}}}}
\newcommand{\OtherTok}[1]{\textcolor[rgb]{0.56,0.35,0.01}{#1}}
\newcommand{\FunctionTok}[1]{\textcolor[rgb]{0.00,0.00,0.00}{#1}}
\newcommand{\VariableTok}[1]{\textcolor[rgb]{0.00,0.00,0.00}{#1}}
\newcommand{\ControlFlowTok}[1]{\textcolor[rgb]{0.13,0.29,0.53}{\textbf{#1}}}
\newcommand{\OperatorTok}[1]{\textcolor[rgb]{0.81,0.36,0.00}{\textbf{#1}}}
\newcommand{\BuiltInTok}[1]{#1}
\newcommand{\ExtensionTok}[1]{#1}
\newcommand{\PreprocessorTok}[1]{\textcolor[rgb]{0.56,0.35,0.01}{\textit{#1}}}
\newcommand{\AttributeTok}[1]{\textcolor[rgb]{0.77,0.63,0.00}{#1}}
\newcommand{\RegionMarkerTok}[1]{#1}
\newcommand{\InformationTok}[1]{\textcolor[rgb]{0.56,0.35,0.01}{\textbf{\textit{#1}}}}
\newcommand{\WarningTok}[1]{\textcolor[rgb]{0.56,0.35,0.01}{\textbf{\textit{#1}}}}
\newcommand{\AlertTok}[1]{\textcolor[rgb]{0.94,0.16,0.16}{#1}}
\newcommand{\ErrorTok}[1]{\textcolor[rgb]{0.64,0.00,0.00}{\textbf{#1}}}
\newcommand{\NormalTok}[1]{#1}
\usepackage{graphicx,grffile}
\makeatletter
\def\maxwidth{\ifdim\Gin@nat@width>\linewidth\linewidth\else\Gin@nat@width\fi}
\def\maxheight{\ifdim\Gin@nat@height>\textheight\textheight\else\Gin@nat@height\fi}
\makeatother
% Scale images if necessary, so that they will not overflow the page
% margins by default, and it is still possible to overwrite the defaults
% using explicit options in \includegraphics[width, height, ...]{}
\setkeys{Gin}{width=\maxwidth,height=\maxheight,keepaspectratio}
\IfFileExists{parskip.sty}{%
\usepackage{parskip}
}{% else
\setlength{\parindent}{0pt}
\setlength{\parskip}{6pt plus 2pt minus 1pt}
}
\setlength{\emergencystretch}{3em}  % prevent overfull lines
\providecommand{\tightlist}{%
  \setlength{\itemsep}{0pt}\setlength{\parskip}{0pt}}
\setcounter{secnumdepth}{0}
% Redefines (sub)paragraphs to behave more like sections
\ifx\paragraph\undefined\else
\let\oldparagraph\paragraph
\renewcommand{\paragraph}[1]{\oldparagraph{#1}\mbox{}}
\fi
\ifx\subparagraph\undefined\else
\let\oldsubparagraph\subparagraph
\renewcommand{\subparagraph}[1]{\oldsubparagraph{#1}\mbox{}}
\fi

%%% Use protect on footnotes to avoid problems with footnotes in titles
\let\rmarkdownfootnote\footnote%
\def\footnote{\protect\rmarkdownfootnote}

%%% Change title format to be more compact
\usepackage{titling}

% Create subtitle command for use in maketitle
\newcommand{\subtitle}[1]{
  \posttitle{
    \begin{center}\large#1\end{center}
    }
}

\setlength{\droptitle}{-2em}

  \title{Capital Bikeshare System Analysis}
    \pretitle{\vspace{\droptitle}\centering\huge}
  \posttitle{\par}
    \author{Ross Bradley}
    \preauthor{\centering\large\emph}
  \postauthor{\par}
      \predate{\centering\large\emph}
  \postdate{\par}
    \date{October 15, 2018}


\begin{document}
\maketitle

\subsection{Introduction}\label{introduction}

\section{Washington D.C. Bike Share
Analysis}\label{washington-d.c.-bike-share-analysis}

``Bike sharing systems are new generation of traditional bike rentals
where whole process from membership, rental and return back has become
automatic. Through these systems, user is able to easily rent a bike
from a particular position and return back at another position.
Currently, there are about over 500 bike-sharing programs around the
world which is composed of over 500 thousands bicycles. Today, there
exists great interest in these systems due to their important role in
traffic, environmental and health issues.

Apart from interesting real world applications of bike sharing systems,
the characteristics of data being generated by these systems make them
attractive for the research. Opposed to other transport services such as
bus or subway, the duration of travel, departure and arrival position is
explicitly recorded in these systems. This feature turns bike sharing
system into a virtual sensor network that can be used for sensing
mobility in the city. Hence, it is expected that most of important
events in the city could be detected via monitoring these data." -
\href{https://archive.ics.uci.edu/ml/datasets/Bike+Sharing+Dataset}{UC -
Irvine}

Below is a proposition to analyze a dataset provided by the Capital
BikeShare system will be the foundation of my Capstone Project for
Springboard's Intro to Data Science course. This dataset provides
information on the number of users, date information, and the weather
details broken down by hour and day. The data was collected between 2011
and 2012 in the Washington D.C. area.

\subsection{Proposition}\label{proposition}

A company needs to know its bottom line. Using the data above, I plan to
predict with confidence the company's average number of users moving
forward. Capital Bikeshare system will find this information important
to know to predict their revenue. With a confident prediction, the
company will have an estimate of how many riders will use their service
and have a baseline of how much revenue they can expect to generate.
Knowing your expected returns allows a company to know how much and when
they can take risks.

The idea is to take the dataset and identify what effect each weather
condition has on rider turnout. Does heavy rain keep people away more
than 90 degree weather? How many registered users are guaranteed to ride
each day? I'm proposing to use the data available to find the variables
that have the strongest effect on user turnout and build a model to
accurately give a count of users based on expected future weather
patterns.

\subsection{Data \& Variables}\label{data-variables}

There are 2 datasets (daily and hourly) we have to analyze and 16
variables in the original sets. We are provided 2 calendar years of D.C.
weather data and the daily or hourly count of users on the company's
bikes on each interval given the dataset. Because we care more about the
overarching picture, our focus will be on the daily dataset.

7 of our variables are in regards to \textbf{time and date}. This
includes the date, the day of the week, if it were a holiday, if it were
a workday, month, season, and year (0 for 2011 and 1 for 2012). 5
variables define the \textbf{weather}: temperature, ``real feel''
temperature, weather situation put into 4 categories based on level of
severity, humididy, and windspeed. Lastly we have 3 variables
categorizing each interval's \textbf{user count}, broken out by number
of registered riders, casual riders, and the sum of the two.

Our output variable and the focus of the project will be the
\textbf{count}. This is the sum of both the registered and casual
columns which gives the total number of riders in the time interval. Our
function will primarily be built off weather and date variables. Our
initial assumption being that weather and date both play a role in
predicting the daily number of riders.

\section{Data wrangling}\label{data-wrangling}

We are lucky to be working with a dataset that only has 1 missing value.
We had a 0 value humidity on 1 date on the daily dataset. Luckily, this
value could be calculated and entered based on the data from the hourly
dataset. No assumptions needed to be made about the missing value since
the formula for how the humidity on the daily dataset was the average of
humidity values for the day in the hourly dataset.

The season, year, month, hour, weekday, workingday, and weather
situation are all collected as integer values between 0 and their
respective maximum levels. We must provide code to R so that these
values are interpreted as categorical instead of numerical numbers. The
season column has had its original values converted from 1-4 to
``Spring'', ``Summer'', etc. for added clarity.

2 additional rows were created, tempC and tempF. The initial temperature
variable is a proportion calculated by the data collector. The
proportion has been reconverted to the real temperature in Celcius and
Fahrenheit for clarity.

\begin{Shaded}
\begin{Highlighting}[]
\CommentTok{#Updating Season Columns}
\KeywordTok{as.character}\NormalTok{(bike.share}\OperatorTok{$}\NormalTok{season,}\DataTypeTok{stringsAsFactors=}\OtherTok{FALSE}\NormalTok{)}
\end{Highlighting}
\end{Shaded}

\begin{Shaded}
\begin{Highlighting}[]
\NormalTok{winter.vec <-}\StringTok{ }\NormalTok{bike.share}\OperatorTok{$}\NormalTok{season }\OperatorTok{==}\StringTok{"1"}
\NormalTok{bike.share}\OperatorTok{$}\NormalTok{season[winter.vec] <-}\StringTok{ "Winter"}

\NormalTok{spring.vec <-}\StringTok{ }\NormalTok{bike.share}\OperatorTok{$}\NormalTok{season }\OperatorTok{==}\StringTok{"2"}
\NormalTok{bike.share}\OperatorTok{$}\NormalTok{season[spring.vec] <-}\StringTok{ "Spring"}

\NormalTok{summer.vec <-}\StringTok{ }\NormalTok{bike.share}\OperatorTok{$}\NormalTok{season }\OperatorTok{==}\StringTok{"3"}
\NormalTok{bike.share}\OperatorTok{$}\NormalTok{season[summer.vec] <-}\StringTok{ "Summer"}

\NormalTok{fall.vec<-}\StringTok{ }\NormalTok{bike.share}\OperatorTok{$}\NormalTok{season }\OperatorTok{==}\StringTok{"4"}
\NormalTok{bike.share}\OperatorTok{$}\NormalTok{season[fall.vec] <-}\StringTok{ "Fall"}


\CommentTok{#Updating temp column to degrees in Fahrenheit}
\NormalTok{temp.form <-}\StringTok{ }\ControlFlowTok{function}\NormalTok{(x)\{}
\NormalTok{  output <-}\StringTok{ }\NormalTok{((x}\OperatorTok{*}\DecValTok{47}\NormalTok{)}\OperatorTok{-}\DecValTok{8}\NormalTok{)}\OperatorTok{*}\NormalTok{(}\DecValTok{9}\OperatorTok{/}\DecValTok{5}\NormalTok{)}\OperatorTok{+}\DecValTok{32}
  \KeywordTok{return}\NormalTok{(output)}
\NormalTok{\}}

\NormalTok{bike.share}\OperatorTok{$}\NormalTok{temp <-}\StringTok{ }\KeywordTok{temp.form}\NormalTok{(}\DataTypeTok{x =}\NormalTok{ bike.share}\OperatorTok{$}\NormalTok{temp)}

\CommentTok{#Updating hum column to percentages}
\NormalTok{temp.hum <-}\StringTok{ }\ControlFlowTok{function}\NormalTok{(x)\{}
\NormalTok{  output <-}\StringTok{ }\NormalTok{(x}\OperatorTok{*}\DecValTok{100}\NormalTok{)}
  \KeywordTok{return}\NormalTok{(output)}
\NormalTok{\}}

\NormalTok{bike.share}\OperatorTok{$}\NormalTok{hum <-}\StringTok{ }\KeywordTok{temp.hum}\NormalTok{(}\DataTypeTok{x =}\NormalTok{ bike.share}\OperatorTok{$}\NormalTok{hum)}
\end{Highlighting}
\end{Shaded}

\section{Data Story}\label{data-story}

We start by looking at some of the direct correlations between the
variables and the count of riders to see which has the most direct
effect on rider count:

\begin{Shaded}
\begin{Highlighting}[]
\NormalTok{tempcorr <-}\StringTok{ }\KeywordTok{cor.test}\NormalTok{(}\DataTypeTok{x =}\NormalTok{ bike.share}\OperatorTok{$}\NormalTok{tempF, }\DataTypeTok{y =}\NormalTok{ bike.share}\OperatorTok{$}\NormalTok{cnt, }\DataTypeTok{method =} \StringTok{"spearman"}\NormalTok{)}
\NormalTok{windcorr <-}\StringTok{ }\KeywordTok{cor.test}\NormalTok{(}\DataTypeTok{x =}\NormalTok{ bike.share}\OperatorTok{$}\NormalTok{windspd, }\DataTypeTok{y =}\NormalTok{ bike.share}\OperatorTok{$}\NormalTok{cnt, }\DataTypeTok{method =} \StringTok{"spearman"}\NormalTok{)}
\NormalTok{humcorr <-}\StringTok{ }\KeywordTok{cor.test}\NormalTok{(}\DataTypeTok{x =}\NormalTok{ bike.share}\OperatorTok{$}\NormalTok{hum, }\DataTypeTok{y =}\NormalTok{ bike.share}\OperatorTok{$}\NormalTok{cnt, }\DataTypeTok{method =} \StringTok{"spearman"}\NormalTok{)}
\NormalTok{sitcorr <-}\StringTok{ }\KeywordTok{cor.test}\NormalTok{(}\DataTypeTok{x =}\NormalTok{ bike.share}\OperatorTok{$}\NormalTok{weathersit, }\DataTypeTok{y =}\NormalTok{ bike.share}\OperatorTok{$}\NormalTok{cnt, }\DataTypeTok{method =} \StringTok{"spearman"}\NormalTok{)}
\end{Highlighting}
\end{Shaded}

\begin{Shaded}
\begin{Highlighting}[]
\NormalTok{windcorr}
\end{Highlighting}
\end{Shaded}

\begin{verbatim}
## 
##  Spearman's rank correlation rho
## 
## data:  bike.share$windspd and bike.share$cnt
## S = 79243000, p-value = 2.969e-09
## alternative hypothesis: true rho is not equal to 0
## sample estimates:
##       rho 
## -0.217197
\end{verbatim}

\begin{Shaded}
\begin{Highlighting}[]
\NormalTok{humcorr}
\end{Highlighting}
\end{Shaded}

\begin{verbatim}
## 
##  Spearman's rank correlation rho
## 
## data:  bike.share$hum and bike.share$cnt
## S = 71888000, p-value = 0.004792
## alternative hypothesis: true rho is not equal to 0
## sample estimates:
##        rho 
## -0.1042218
\end{verbatim}

\begin{Shaded}
\begin{Highlighting}[]
\NormalTok{tempcorr}
\end{Highlighting}
\end{Shaded}

\begin{verbatim}
## 
##  Spearman's rank correlation rho
## 
## data:  bike.share$tempF and bike.share$cnt
## S = 24607000, p-value < 2.2e-16
## alternative hypothesis: true rho is not equal to 0
## sample estimates:
##       rho 
## 0.6220345
\end{verbatim}

\begin{Shaded}
\begin{Highlighting}[]
\NormalTok{sitcorr}
\end{Highlighting}
\end{Shaded}

\begin{verbatim}
## 
##  Spearman's rank correlation rho
## 
## data:  bike.share$weathersit and bike.share$cnt
## S = 82817000, p-value = 7.111e-14
## alternative hypothesis: true rho is not equal to 0
## sample estimates:
##        rho 
## -0.2720974
\end{verbatim}

One can see that temperature has the strongest and only seemingly
positive correlation. The rest are negative. A closer look at the
visualizations of the data may prove helpful in understanding their
relationship.

\begin{Shaded}
\begin{Highlighting}[]
\KeywordTok{library}\NormalTok{(ggplot2)}
\KeywordTok{library}\NormalTok{(ggpmisc)}
\CommentTok{#Total Riders by humidity.}
\CommentTok{#shows slight negative trend. High temperature is above the line, cool is below.}

\KeywordTok{ggplot}\NormalTok{(bike.share, }\KeywordTok{aes}\NormalTok{(}\DataTypeTok{x=}\NormalTok{hum, }\DataTypeTok{y=}\NormalTok{cnt, }\DataTypeTok{col =}\NormalTok{ temp))}\OperatorTok{+}
\StringTok{  }\KeywordTok{geom_point}\NormalTok{()}\OperatorTok{+}
\StringTok{  }\KeywordTok{labs}\NormalTok{(}\DataTypeTok{x=}\StringTok{"Humidity (%)"}\NormalTok{, }\DataTypeTok{y =} \StringTok{"Total Riders"}\NormalTok{)}\OperatorTok{+}
\StringTok{  }\KeywordTok{scale_color_gradientn}\NormalTok{(}\DataTypeTok{colors =} \KeywordTok{brewer.pal}\NormalTok{(}\DecValTok{9}\NormalTok{, }\StringTok{"YlOrRd"}\NormalTok{))}\OperatorTok{+}
\StringTok{  }\CommentTok{#facet_grid(weathersit~.)+}
\StringTok{  }\KeywordTok{geom_smooth}\NormalTok{(}\DataTypeTok{method =} \StringTok{"lm"}\NormalTok{, }\DataTypeTok{formula =}\NormalTok{ y }\OperatorTok{~}\StringTok{ }\NormalTok{x)}
\end{Highlighting}
\end{Shaded}

\includegraphics{Capstone_Markdown_files/figure-latex/unnamed-chunk-5-1.pdf}
The chart above shows that humidity trends slightly negatively with
total riders. Colored by temperature, here we can see that high
temperatures are above our trend line and lower temperatures are mostly
below. This could mean that temperature correlates positively with rider
count.

Let's continue to review temperature as a variable to predict total
riders in a day. The below chart will plot total riders in a given day.
We think that temperature will play a role, but but also break down the
points by season.

\begin{Shaded}
\begin{Highlighting}[]
\CommentTok{#Total Riders by temperature, following a log function of the data. }
\CommentTok{#Slight trend noticed that very high T is below the trend line}
\CommentTok{# y = -167 + 78.5x}
\KeywordTok{library}\NormalTok{(ggpmisc)}
\KeywordTok{ggplot}\NormalTok{(bike.share, }\KeywordTok{aes}\NormalTok{(}\DataTypeTok{x=}\NormalTok{temp, }\DataTypeTok{y=}\NormalTok{cnt, }\DataTypeTok{col =}\NormalTok{ hum))}\OperatorTok{+}
\StringTok{  }\KeywordTok{labs}\NormalTok{(}\DataTypeTok{x=}\StringTok{"Temp (F)"}\NormalTok{, }\DataTypeTok{y =} \StringTok{"Total Riders"}\NormalTok{)}\OperatorTok{+}
\StringTok{  }\KeywordTok{geom_smooth}\NormalTok{(}\DataTypeTok{method =} \StringTok{"lm"}\NormalTok{, }\DataTypeTok{formula =}\NormalTok{ y}\OperatorTok{~}\NormalTok{x)}\OperatorTok{+}
\StringTok{  }\KeywordTok{stat_poly_eq}\NormalTok{(}\DataTypeTok{parse =}\NormalTok{T, }\KeywordTok{aes}\NormalTok{(}\DataTypeTok{label=}\NormalTok{ ..eq.label..), }\DataTypeTok{formula =}\NormalTok{ y}\OperatorTok{~}\NormalTok{x)}
\end{Highlighting}
\end{Shaded}

\includegraphics{Capstone_Markdown_files/figure-latex/unnamed-chunk-6-1.pdf}
We can see that temperature tends to trend upwards, but we know a 1:1
correlation can't exist. Eventually it will get too hot for bike
sharing! Because we know this isn't a perfect relationship, let's look
at the data given a cubic function to estimate the outcome. We also want
to include the weather situation in these visualizations to see how
harsher weather can effect the rider count. We would suspect that level
1 would be higher up where level 3 would yield less riders.

\begin{Shaded}
\begin{Highlighting}[]
\CommentTok{#Same as above but with a cubic function showing that as temperature increases too much we have a negative trend!}
\CommentTok{# y = 1460 -153x + 6.51x^2 - .0492x^3}
\NormalTok{formula <-}\StringTok{ }\NormalTok{y }\OperatorTok{~}\StringTok{ }\KeywordTok{poly}\NormalTok{(x,}\DecValTok{3}\NormalTok{, }\DataTypeTok{raw =} \OtherTok{TRUE}\NormalTok{)}

\KeywordTok{ggplot}\NormalTok{(bike.share, }\KeywordTok{aes}\NormalTok{(}\DataTypeTok{x=}\NormalTok{temp, }\DataTypeTok{y=}\NormalTok{cnt, }\DataTypeTok{col =}\NormalTok{ weathersit))}\OperatorTok{+}
\StringTok{  }\KeywordTok{geom_point}\NormalTok{(}\KeywordTok{aes}\NormalTok{(}\DataTypeTok{color =} \KeywordTok{factor}\NormalTok{(weathersit)))}\OperatorTok{+}
\StringTok{  }\KeywordTok{labs}\NormalTok{(}\DataTypeTok{x=}\StringTok{"Temperature (F)"}\NormalTok{, }\DataTypeTok{y =} \StringTok{"Total Riders"}\NormalTok{)}\OperatorTok{+}
\StringTok{  }\CommentTok{#scale_color_gradientn(colors = brewer.pal(9, "YlOrRd"))+}
\StringTok{  }\CommentTok{#facet_grid(weathersit~.)+}
\StringTok{  }\KeywordTok{geom_smooth}\NormalTok{(}\DataTypeTok{method =} \StringTok{"glm"}\NormalTok{, }\DataTypeTok{formula =}\NormalTok{ formula)}\OperatorTok{+}
\StringTok{  }\KeywordTok{stat_poly_eq}\NormalTok{(}\DataTypeTok{parse =}\NormalTok{T, }\KeywordTok{aes}\NormalTok{(}\DataTypeTok{label=}\NormalTok{ ..eq.label..), }\DataTypeTok{formula =}\NormalTok{ formula)}
\end{Highlighting}
\end{Shaded}

\includegraphics{Capstone_Markdown_files/figure-latex/unnamed-chunk-7-1.pdf}
From the above chart we see that after the temperature reaches
\textasciitilde{}75 degrees our total number of bikers tends to
decrease. We can also see that weather categories 1 and 2 are both above
and below the line of best fit. Category 3 is under the curve in all
instances which is what we expected.

What is very interesting is that even with a cubic function, the data is
very far parsed. There must be more variables that come into play when
people choose to go ride. Let's investigate further.

Next we will visualize the average wind speed of the day against the
total rider count.

\begin{Shaded}
\begin{Highlighting}[]
\CommentTok{#windspeed verse count}
\KeywordTok{ggplot}\NormalTok{(bike.share, }\KeywordTok{aes}\NormalTok{(}\DataTypeTok{x=}\NormalTok{bike.share}\OperatorTok{$}\NormalTok{windspd, }\DataTypeTok{y =}\NormalTok{ cnt))}\OperatorTok{+}
\StringTok{  }\KeywordTok{geom_point}\NormalTok{()}\OperatorTok{+}
\StringTok{  }\KeywordTok{geom_smooth}\NormalTok{(}\DataTypeTok{method =} \StringTok{"lm"}\NormalTok{, }\DataTypeTok{formula =}\NormalTok{ y}\OperatorTok{~}\NormalTok{x)}\OperatorTok{+}
\StringTok{  }\KeywordTok{theme}\NormalTok{(}\DataTypeTok{panel.background =} \KeywordTok{element_rect}\NormalTok{(}\DataTypeTok{fill =} \StringTok{"white"}\NormalTok{),}
        \DataTypeTok{axis.line.x=}\KeywordTok{element_line}\NormalTok{(),}
        \DataTypeTok{axis.line.y=}\KeywordTok{element_line}\NormalTok{()) }\OperatorTok{+}
\StringTok{  }\KeywordTok{ggtitle}\NormalTok{(}\StringTok{"Linear Model Fitted to Data"}\NormalTok{)}
\end{Highlighting}
\end{Shaded}

\includegraphics{Capstone_Markdown_files/figure-latex/unnamed-chunk-8-1.pdf}

Here we see a slight negative trend here as well. Temperature seems to
be our best indicator but if we combine these variables together we may
have a clear picture.

\section{Ideas for the future}\label{ideas-for-the-future}

A deeper dive into the data could have shown a more clear predictor of
the data.

\section{Conclusion}\label{conclusion}


\end{document}
